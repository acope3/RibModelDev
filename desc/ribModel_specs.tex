\documentclass[letter,10pt]{article}
\usepackage[utf8]{inputenc}
\usepackage[margin=1.0in]{geometry}
\usepackage{setspace}
\usepackage{listings}
\usepackage{color}
\usepackage[noend]{algpseudocode}
\usepackage{csquotes}

\definecolor{dkgreen}{rgb}{0,0.6,0}
\definecolor{gray}{rgb}{0.5,0.5,0.5}
\definecolor{mauve}{rgb}{0.58,0,0.82}

\lstset{frame=tb,
  language=C++,
  aboveskip=3mm,
  belowskip=3mm,
  showstringspaces=false,
  columns=flexible,
  basicstyle={\small\ttfamily},
  numbers=none,
  numberstyle=\tiny\color{gray},
  keywordstyle=\color{blue},
  commentstyle=\color{dkgreen},
  stringstyle=\color{mauve},
  breaklines=true,
  breakatwhitespace=true,
  tabsize=3
}

%%%%%%%%%%%%%%% BEGIN LOCAL COMMANDS %%%%%%%%%%%%%%%%%%%
\newcommand{\sep}{\discretionary{}{}{}} % Used to help with text separation, hboxes.
%%%%%%%%%%%%%%% END LOCAL COMMANDS %%%%%%%%%%%%%%%%%%%

%opening
\title{ribModel Code specifications}


\begin{document}

\maketitle
\newpage
\tableofcontents
\newpage

\section{C++ Code}

\subsection{Bracket Format}
\begin{itemize}
    \item Brackets should begin on the line below a function, if statement, while clause, etc.
    \item
    One-line statements may have brackets omitted with just an indentation, or it may
    be put on the same line.
    \item Example code:
    \begin{lstlisting}
        void Gene::cleanSeq()
        {
            std::string valid = "ACGTN";
            for (unsigned i = 0; i < seq.length(); i++)
            {
                if (valid.find(seq[i]) == std::string::npos)
                    seq.erase(i);
            }
        }
    \end{lstlisting}
\end{itemize}

\subsection{Commenting}
\begin{itemize}
    \item
        There should be a documentation block above each function. The format of the documentation
        block is a long-form C++ comment with the name of the function and whether or not it is
        exposed to RCPP, followed by *-separated lines listing the arguments and any further
        description.
    \item Caveats to the function and any TODOs are listed in this documentation block at the end.
    \item If you copy code, mention the source so the code can be double checked-later.
    \item Example code:
    \begin{lstlisting}
        /* Gene = operator (NOT EXPOSED)
         * Arguments: Gene object
         * Overloaded definition of the assignment operator ( = ). Function is
         * similar to that of the copy constructor.
        */
    \end{lstlisting}
    \item Within the code itself, document/describe parts of code/functions 
    that are not self-explanatory
    (Can you come back to it a month later and still know what the code does?).
    \item Major blocks of code separating divisions in many functions' usage or unit testing
    may be denoted by \verb+//--------------------//+ blocks.
    \item Example code:
    \begin{lstlisting}
            //---------------------------------------//
            //------ initParameterSet Function ------//
            //---------------------------------------//
    \end{lstlisting}
\end{itemize}

\subsection{Code Spacing}
\begin{itemize}
    \item
    Between different functions and their respective ends and comment blocks, 
    there should be two lines of spacing between them.
    \item 
    Before a major section of code (marked by \verb+//--------------------//+ blocks)
    there should be five lines of spacing at the end of the previous function.
    \item After that comment block, however, there may be only two lines of spacing like usual.
\end{itemize}

\subsection{Function/Variable Naming Conventions}
\begin{itemize}
    \item Functions should be named as clearly as possible, in camelCase.
    \item Example: \enquote{Gene::getObservedSynthesisRateValues()}
    \item Variables that are meant to be temporary should be named as such.
    \item Example: \enquote{tmpGene}
    \item Use speaking variable names.
\end{itemize}

\subsection{New Functions/Classes}
\begin{itemize}
    \item Create a test case for new functions in the class Testing.
    \item Ensure that constructors create a VALID object. 
    That does not necessary mean all information has to be placed, but all members have to be properly initialized. 
    \item Functions are only defined in headers, but never implemented.
    \item New classes should follow naming convention: [MODEL NAME]\sep Parameter, [MODEL NAME]\sep Model.
    That should be generally applied for inherited classes.
    \item Constructors can not have more than 6 arguments. 
    That causes us to use setter functions in cases where more arguments are needed. 
    Adjustment can be made once Rcpp can handle more arguments.
\end{itemize}

\subsection{Implementation}
\begin{itemize}
    \item Avoid infinite loop structures ( for(;;), while(true), ... ) that break under various conditions.
    \item Avoid dynamically-allocated arrays and use vectors instead if possible.
    Dynamic allocated arrays seem to cause problems with openmp.
    \item All log output should use the wrapper functions \enquote{my\sep \_print}, \enquote{my\sep \_printError}, \enquote{my\sep \_printWarning}.
    \item Rcpp modules are used to expose classes and member functions. Member functions and static functions are exposed using a different syntax (see code). 
\end{itemize}

\section{R Code}
\begin{itemize}
    \item Stick to the Google style guide for R.
\end{itemize}

\end{document}

